\documentclass[%
	%draft,
	paper=a4,%
	twoside=true,%
	abstract=true]{scrartcl}

\usepackage[utf8]{inputenc}
\usepackage[english]{babel}
\usepackage{graphicx}	
\usepackage{ifpdf}
\ifpdf
	\usepackage{lineno}
		\linenumbers\modulolinenumbers[2]
	\usepackage{microtype}
	\usepackage[colorinlistoftodos,shadow]{todonotes}	
\else
	\DeclareGraphicsExtensions{.png}
	\usepackage{setspace}
		\onehalfspacing
	\usepackage[disable]{todonotes}		
\fi	
\usepackage[load-configurations=binary]{siunitx}
	\DeclareSIUnit\Molar{\textsc{m}}
\usepackage{svn-multi}
\usepackage{subfig}
\usepackage{booktabs}
\usepackage{fancyhdr}
\usepackage{tikz}
\usepackage{pgfplots}
\usepackage[numbers,square,sort&compress]{natbib}
\usepackage{scrtime}
\usepackage[version=3]{mhchem}
\usepackage{lastpage}	
\usepackage{xspace}
\usepackage[autostyle=true]{csquotes}
\usepackage[backref]{hyperref}

% Subversion Information
\svnidlong
{$HeadURL$}
{$LastChangedDate$}
{$LastChangedRevision$}
{$LastChangedBy$}
\svnid{$Id$} 
 
\pagestyle{fancy}
\fancyfoot{}
\fancyfoot[OR]{\tiny Typeset on \today\ at \thistime\ from \href{\svnkw{HeadURL}}{SVN-Version \svnkw{LastChangedRevision}} | Page \thepage\ of \pageref{LastPage}}
\fancyfoot[EL]{\tiny Page \thepage\ of \pageref{LastPage} | Typeset on \today\ at \thistime\ from \href{\svnkw{HeadURL}}{SVN-Version \svnkw{LastChangedRevision}}}
 
\newcommand{\imsize}{\linewidth}
\newlength\imagewidth           % needed for scalebars
\newlength\imagescale           % ditto

\newcommand{\footremember}[2]{\footnote{#2}\newcounter{#1}\setcounter{#1}{\value{footnote}}}
\newcommand{\footrecall}[1]{\footnotemark[\value{#1}]}

\newcommand{\superscript}[1]{\ensuremath{^{\textrm{#1}}}}
\newcommand{\subscript}[1]{\ensuremath{_{\textrm{#1}}}}

\newcommand{\ie}{i.\,e.\xspace}
\newcommand{\eg}{e.\,g.\xspace}
\newcommand{\twod}{2\textsc{d}\xspace}
\newcommand{\threed}{3\textsc{d}\xspace}

\newcommand{\todomarco}[2][]{\todo[color=cyan!62!white,, #1]{Marco: #2}}
\newcommand{\todojcs}[2][]{\todo[color=magenta!62!white, #1]{Johannes: #2}}
\newcommand{\todome}[2][]{\todo[color=yellow!62!white, #1]{David: #2}}

\newcommand{\subfigureautorefname}{\figureautorefname} % make \autoref work with \subfloat
 
\title{Running Title: Acinar growth over lung development}
\subtitle{Typeset on \today\ at \thistime\ from Rev \svnkw{LastChangedRevision} (\svnday.\svnmonth.\svnyear\ \svnhour:\svnminute)}

\author{%
	David Haberthür\footremember{ana}{Institute of Anatomy, University of Bern, Switzerland}%
	\and Sébastien Barré\footrecall{ana}%
	\and Lilian Salm\footrecall{ana}%
	\and Marco Stampanoni\footremember{psi}{Swiss Light Source, Paul Scherrer Institut, Villigen, Switzerland}\ \footremember{eth}{Institute for Biomedical Engineering, Swiss Federal Institute of Technology and University of Zürich, Switzerland}%
	\and Johannes C. Schittny\footrecall{ana}%
	}
\date{}

\begin{document}
\renewcommand{\subsectionautorefname}{\sectionautorefname}
\renewcommand{\subsubsectionautorefname}{\sectionautorefname}
\maketitle

\begin{abstract}
Here be the abstract, including points like:
\begin{itemize}
	\item Lung development
	\item Acinar Volume
	\item Gold Standard STEPanizer proves MeVisLab, obviously
	\item Opens the possibility to study postnatal development of functional lung units, only possible in \threed
\end{itemize}
\end{abstract}
\listoftodos
\clearpage

\section{Introduction}\label{sec:Introduction}
Due a restricted availability of high resolution three-dimensional (\threed) imaging methods the knowledge about the development of the functional unit of the lung is limited. These functional units of the lung parenchyma are the so called pulmonary acini, which correspond to the gas-exchange volume in the lung which is ventilated by one purely conducting airway~\cite{Rodriguez1987}, see \autoref{fig:lung schematic}.

Using synchrotron radiation based tomographic microscopy with enhanced field of view \cite{Haberthuer2010a} we developed a method to evaluate the volume of single acini throughout postnatal lung development. In this manuscript we present the methodology to extract single acini in high-resolution three-dimensional datasets obtained with synchrotron-radiation based tomographic microscopy.

Up to now it has not been possible to extract a large amount of single acini from physical sections of lung tissue. \citet{Woodward2005} achieved three-dimensional reconstructions of three adult ducks using manual effort to trace serial sections of the tissue. Since classic histological preparations (like cutting in a Microtome, serial sectioning, etc.) cause deformations in the sectioned tissue like compression artifacts from cutting and folding or shearing of the tissue during mounting. This essentially destroys the three-dimensional information inherent in successive slices of the lung structure. Such a destruction makes it impossible to observe and trace single acini to extract their volume. Tomographic imaging of the lung tissue preserves this three-dimensional structure, since the slices through the tissue are entirely virtual and thus makes it possible to observe the volume of single acini.

Through careful segmentation we can then extract single functional lung units---the so-called acini---and analyze those with classic and accepted methods in both two and three dimensions. In addition to three-dimensional volume analysis in a visualization software we used standard stereological analysis \cite{Hsia2010} as a gold-standard to compare our results with a ground truth. As stated by \citet{Hsia20110}, Stereology refers to the mathematical methods for analyzing these properties of an irregular three-dimensional structure using two-dimensional planar sections obtained by physical or optical imaging techniques.

Extracting single functional lung units from the tomographic datasets enabled us to perform such a stereological analysis to obtain a complete description of the functional lung units including alveolar number, surface and volume and compare these values over the course of the postnatal lung development. Such a complete description is only possible with tomographic data, since the three-dimensional information in the sample is not destroyed.

\subsection{Lung development}
As \citet{Schittny2007a} described, descriptions of the stages of lung development are based on light microscopical observations of morphological changes in the developing lung. In this manuscript, we present a method to analyze and describe the postnatal development of the functional units of the lung, \ie changes in their volume and surface.

\subsection{Lung structure and the functional units of the lung}
The airway structure of the mammalian lung is formed from dichotomous branches\todojcs{Or only human and not general mammalian lung?}, starting from the trachea. The first branching generations lead into the bronchi (see \autoref{fig:lung schematic}). With increasing depth into the airway tree, the airway diameter of the airways is reduced, the bronchi are divided into bronchioles, leading to the terminal bronchioles which mark the end of the pipe-like purely conducting airways. The respiratory bronchioles mark the start of the gas-exchange region in the lung. We start to see changes in the airway wall structure--alveolar outpouchings---which mark the change between purely conducting and gas-exchanging airways (see \autoref{fig:ManholeCoverExplanation}). After this point, the so-called acinar airways and the acinus---the functional unit of the lung---begin. The observation of such changes in the airway wall make it possible to extract single acini from the three-dimensional tomographic datasets, as described later in \autoref{sec:manholecovers}.

\renewcommand{\imsize}{.618\linewidth}
\begin{figure}
	\centering
	\pgfmathsetlength{\imagewidth}{\imsize}%
	\pgfmathsetlength{\imagescale}{\imagewidth/1635}%
	\def\x{1010}% scalebar-x at golden ratio of x=1635px
	\def\y{1831}% scalebar-y at 90% of height of y=2034px
	\def\shadow{10}% shadow parameter for scalebar
	\begin{tikzpicture}[x=\imagescale,y=-\imagescale]
		\clip (0,0) rectangle (1635,2034);
		\node[anchor=north west, inner sep=0pt, outer sep=0pt] at (0,0) {\includegraphics[width=\imagewidth]{img/AcinusDiagram/Schittny2007a_edited.png}};
		\node at (810,75) [right] {Trachea};
		\node at (180,575) [right, rotate=28] {Pleura};
		\node at (325,650) [right] {Bronchi};
		\node at (515,1100) [left] {Bronchioli};
		\node at (700,1450) [left] {transitory Bronchioles};
		\node at (700,1700) [left] {Alveoli};
		\draw [|->,ultra thick] (1550,1329) -- (1550,400) node [midway,above,rotate=90] {Conducting Airways};
		\draw [|->,ultra thick] (1550,1349) -- (1550,1700) node [midway,above,rotate=90] {Gas-exchange};% fill=white,semitransparent,text opacity=1
	\end{tikzpicture}%
	\caption{Schematic of the lung structure. The stages of one lung lobe are shown. On average rat lungs have ten generations of Bronchi and four generations of Bronchioli, which together form the conducting airways. The gas-exchanging region starts with one generation of transitory bronchioles which merge into alveolar ducts and end in the alveoli. Figure adapted from \cite{Schittny2007a}.}
	\label{fig:lung schematic}%
	\todo[inline, caption={Copyright release?}]{Copyright release for this figure?}
	\todo[inline, caption={Number of generations?}]{How many generations of alveolar ducts are there in the rat? Just one?}
\end{figure}

The lung structure can be assessed using stereology~\cite{Hsia2010}, as shown by \citet{Tschanz2002}. Such an analysis is generally based on serial sections of the sample, thus the extracted information is a two-dimensional description \blockquote[\cite{Tschanz2002}]{of the parenchymal air space geometry and, due to geometric laws, it is not allowed to extrapolate these \twod statements directly to \threed structures}. With stereological methods it is possible to extract global volume information, but it not easily possible to extract such information from a functional subunit of an organ like the acini in the lung, since it cannot easily be judged which detail on one microscopy slide belongs to which functional unit in the three-dimensional compound.

\section{Materials \& Methods}\label{sec:MM}
\subsection{Rat lung samples}
Results shown in this manuscript have been obtained from lungs from adult Sprague-Dawley rats at day 60 after birth. Animals were deeply anesthetized and their lungs instilled with \SI{2.5}{\percent} glutaraldehyde (\cf{CH2(CH2CHO)2}) in \SI{0.03}{\Molar} potassium-phosphate buffer (pH 7.4) at a constant pressure of \SI{20}{\centi\meter} water column. At this applied pressure, the rat lung reaches its mid-respiratory volume. The instillation was performed via tracheotomy after opening the chest cavity and setting a pneumothorax through perforation of the diaphragm. After instillation the lung was removed from the chest cavity and the instillation pressure was maintained during fixation (in the same fixative at \SI{4}{\celsius} for at least \SI{24}{\hour}) in order to prevent a recoiling of the lung. Details of the whole procedure have been described by \citet{Tschanz2002} as well as \citet{Burri1974}.

After fixation, the samples were prepared for tomographic imaging by postfixation with \SI{1}{\percent} osmium tetroxide (\cf{OsO4}) and staining with \SI{4}{\percent} uranyl nitrate (\cf{UO2(NO3)2}) to increase the x-ray absorption contrast. Using Histoclear (Merck KGaA, Darmstadt, Germany) as an intermedium the samples were then dehydrated in a graded series of ethanol and embedded in paraffin prior to mounting them onto standard scanning electron microscopy sample holders (PLANO GmbH, Wetzlar, Germany) with paraffin~\cite{Tsuda2008}.

\subsection{Tomographic data acquisition}\label{sec:tomcat}
The tomographic experiments were performed at the \href{http://www.psi.ch/sls/tomcat/}{TOMCAT beamline} at the \href{http://www.psi.ch/sls/}{Swiss Light Source}, \href{http://www.psi.ch/}{Paul Scherrer Institut}, Villigen, Switzerland~\cite{Stampanoni2006a}. The samples were scanned at \SI{20.0}{\kilo\electronvolt} corresponding to a wavelength of \(\lambda=\SI{0.62}{\angstrom}\). % \lambda = 1.24e-6 eV/m / 20 keV = 6.197796e-11 m = 0.6197796 Ångström
After penetration through the sample, the x-rays were converted into visible light by a \SI{20}{\micro\meter} thick LuAG:Ce scintillator (Lutetium Aluminum Garnet activated by cerium, \cf{Lu3Al5O12}, \href{http://www.crytur.cz/}{Crytur Ltd.}, Turnov, Czech Republic).

A 10\(\times\) magnifying, diffraction limited microscope optics was used to magnify the image onto a \SI{18}{\micro\meter} thick Cerium doped Yttrium Aluminium Garnet (YAG) (\cf{Y3Al5O12}) scintillator screen. Subsequently, a high-resolution 2048\(\times\)2048 pixel CCD camera (\href{http://www.pco.de/sensitive-cameras/pco2000/}{pco.2000}, \href{http://www.pco.de/}{PCO AG}, Kelheim, Germany) with \SI{14}{\bit} dynamic range was used to digitize this magnified image. To reduce imaging noise and increase the acquisition speed, we operated the detector in 2\(\times\)2 binning mode. As a result, the pixel size was \SI{1.48}{\micro\meter} and the exposure time was between \SIrange{160}{180}{\milli\second}. Details of the TOMCAT beamline setup have been thoroughly described by \citet{Stampanoni2006a}.

To be able to safely distinguish the alveolar septa (approx.~\SIrange{3}{8}{\micro\meter} thick\todojcs{Correct?}) the tomographic images used for analysis of the functional units of the lung need to have a resolution in the order of one to two microns. Since we selectively wanted to extract a large amount of single acini, we not only needed to acquire high resolution tomographic scans, but also acquire dataset with both large volume and in high resolution. Usually---with classic microscopy based imaging methods---a large field of view can only be acquired with low magnification and vice-versa. Since the our samples were larger than the classic field of view of TOMCAT at the aforementioned optical properties (\(1.52\times1.52\times\SI{1.52}{\milli\meter}\)) we would not have been able to image the full volume of our samples. To overcome this problem, we enhanced the enhance the field of view of the TOMCAT beamline at the chosen optical configuration. This enhancement was done with the so called wide-field scanning protocol, described by \citet{Haberthuer2010}.

Briefly summarized, the wide-field scanning process merges three or five independently acquired partial tomographic scans which successively cover the total desired field of view. The single projections of the are merged to one large projection spanning the desired field of view on during tomographic reconstruction. This merging is performed on the tomographic reconstruction cluster of seven \SI{64}{\bit} Opteron machines with four cores and \SI{8}{\giga\byte} RAM each. Implementing the merging on the cluster made it possible to use the classic tomographic reconstruction work flow at TOMCAT, which has been described in detail by \citet{Hintermueller2010}.

Using this wide field scanning approach we increased the lateral field of view at TOMCAT three-fold, while keeping both voxel size and reconstruction quality on the desired level and avoiding the aforementioned trade-off between voxel size and sample volume. Since we also performed a so called stacked scan in relation to the z-axis of the tomographic setup, we even further increased the field of view and the recorded volumes of our samples. We acquired tomographic dataset of approximately 3000\(\times\)3000\(\times\)3072 pixels with \SI{1.48}{\micro\meter} pixel size. This corresponds to a 9-fold increase in recorded volume as compared to a classic scan at TOMCAT, giving us the advantage of both high resolution images and large visible sample volume to extract the single functional lung units.

\subsection{Visualization and Extraction of Acini}
The tomographic datasets of the sample were three-dimensionally analyzed and visualized using \href{http://mevislab.de}{MeVisLab} (Version 2.1 (2010-07-26 Release)~\cite{Bitter2007}, MeVis Medical Solutions AG and Fraunhofer MEVIS - Institute for Medical Image Computing, Bremen, Germany). The calculations have been performed on a Dell Precision T7500 work station (\SI{24}{\giga\byte} RAM, Intel Xeon CPU X5550 at \SI{2.66}{\giga\hertz}, Windows 7 Professional \SI{64}{\bit}).

\subsubsection{Preprocessing}
The tomographic datasets obtained at TOMCAT were converted from a stack of TIFF-files to the native GVR format of MeVisLab, a multi-resolution \href{https://secure.wikimedia.org/wikipedia/en/w/index.php?title=Octree&oldid=409131920}{octree}-based image format. This permitted us to easily switch between resolutions in the dataset to interactively perform the visualization and preliminary analysis on a lower resolution prior to the final analysis on full resolution datasets.

\subsubsection{Manhole Covers}\label{sec:manholecovers}
On binned datasets we extracted conducting airway segments (see \autoref{fig:lung schematic}) using a threshold interval based region growing algorithm~\cite{Zucker1976}. A seed point for the region growing algorithm was manually defined inside the terminal bronchiole or alveolar duct on one of the most proximal slices of the dataset (shown in \autoref{subfig:sample}). This approach made it possible to extract a connected airway segment, which was then further subdivided into conducting and gas-exchanging airways. This division was performed with manually placed circular segmentations stoppers, which we dubbed manhole covers (shown as red discs in \autoref{subfig:airway segment} and \subref*{subfig:extracted acini}). The manhole covers were manually placed in the airway segment based on morphological criteria, \ie on changes in the airway wall structure which mark the entrance point into the acinus.

\renewcommand{\imsize}{\linewidth}
\begin{figure}
	\centering
	\pgfmathsetlength{\imagewidth}{\imsize}%
	\pgfmathsetlength{\imagescale}{\imagewidth/992}%
	\def\x{613}% scalebar-x at golden ratio of x=992px
	\def\y{893}% scalebar-y at 90% of height of y=992px
	\begin{tikzpicture}[x=\imagescale,y=-\imagescale]
		\clip (0,0) rectangle (992,992);
		\node[anchor=north west, inner sep=0pt, outer sep=0pt] at (0,0) {\includegraphics[width=\imagewidth]{img/ManholeCoverExplanation/SliceSomeWhereAtTheBottom_ScaleBar}};
		% 493px = 0.8mm > 100px = 162um > 308px = 500um, 62px = 100um
		\draw[|-|,blue, thick] (524,289) -- (524,781) node [sloped,midway,above,fill=white,semitransparent,text opacity=1] {\SI{0.8}{\milli\meter} (800px) TEMPORARY!};
		\draw[|-|,white, thick] (\x,\y) -- (\x+308,\y) node [midway, above] {\SI{500}{\micro\meter}};
		\draw[yellow,dashed,thick] (409,420) circle (40);
		\draw[yellow,dashed,thick] (357,382) circle (30);
		\draw[yellow,dashed,thick] (265,438) circle (30);
		\draw[yellow,dashed,thick] (297,466) circle (30);						
	\end{tikzpicture}%
	\caption{Slice showing extracted conducting airway (green) and manhole cover (red). The dashed yellow circles highlight some examples of outpouchings of the airway wall which mark the change from conducting to gas-exchanging regions. None of these outpouchings are visible in the airway wall around the green segment. This makes it possible to semiautomatically placed the manhole covers at the regions where the conducting airways change into gas-exchanging airways.}
	\label{fig:ManholeCoverExplanation}
\end{figure}

Using a separate region growing module, we subsequently extracted single acini from the datasets (shown as yellow volumes in \autoref{subfig:extracted acini}) automatically. The manhole covers\footnote{The manhole covers were implemented through a \href{http://www.mevis-research.de/cgi-bin/discus/board-auth.cgi?lm=1282233250&file=/839/11760.html}{custom module (\emph{XMarkerClipPlanes})} written by Milo Hindennach, a member of the MeVis Developer team.} placed in the first step are fully defined through their diameter and \href{https://secure.wikimedia.org/wikipedia/en/w/index.php?title=Surface_normal&oldid=411684319}{surface normal}. This made it possible to automatically define the seed point for additional region growing modules used for the extraction of the single acini. We simply flipped the direction of the surface normal and placed the seed point along this vector slightly behind the manhole cover, inside the acinar airspace. The only manual work left was the iterative selection of the correct gray value threshold used for the region growing segmentation.

\renewcommand{\imsize}{.333\linewidth}
\pgfmathsetlength{\imagewidth}{\imsize}%
\pgfmathsetlength{\imagescale}{\imagewidth/1008}%
\def\x{50}% scalebar-x at golden ratio of x=1008px
\def\y{916+20}% scalebar-y at 90% of height of y=1018px
\begin{figure}
	\centering
	\subfloat[Sample]{%
		\begin{tikzpicture}[x=\imagescale,y=-\imagescale]
			\node[anchor=north west, inner sep=0pt, outer sep=0pt] at (0,0) {\includegraphics[width=\imagewidth]{img/ManholeCover/R108C60B_2010c_Acinus_Sample}};
			% 774px = 4.363mm > 100px = 564um > 89px = 500um, 18px = 100um
			%\draw[|-|,blue,thick] (775,990) -- (33,771) node [sloped,midway,above,fill=white,semitransparent,text opacity=1] {\SI{4.363}{\milli\meter} (2948px) TEMPORARY!};
			\draw[|-|, thick] (\x,\y) -- (\x+89,\y) node [right] {\SI{500}{\micro\meter}};
		\end{tikzpicture}%
		\label{subfig:sample}%
		}%
	\subfloat[Extracted conducting airways with manhole covers shown overlayed over Sample.]{%
		\begin{tikzpicture}[x=\imagescale,y=-\imagescale]
			\node[anchor=north west, inner sep=0pt, outer sep=0pt] at (0,0) {\includegraphics[width=\imagewidth]{img/ManholeCover/R108C60B_2010c_Acinus_Airspace_Overlay}};
			\draw[|-|, thick] (\x,\y) -- (\x+89,\y) node [right] {\SI{500}{\micro\meter}};
		\end{tikzpicture}%
		\label{subfig:airway segment}%
		}%
	\subfloat[Extracted acini.]{%
		\begin{tikzpicture}[x=\imagescale,y=-\imagescale]
			\node[anchor=north west, inner sep=0pt, outer sep=0pt] at (0,0) {\includegraphics[width=\imagewidth]{img/ManholeCover/R108C60B_2010c_Acinus_OverlayNonTransparent}};
			% 774px = 4.363mm > 100px = 564um > 89px = 500um, 18px = 100um
			\draw[|-|, thick] (\x,\y) -- (\x+89,\y) node [right] {\SI{500}{\micro\meter}};
		\end{tikzpicture}%
		\label{subfig:extracted acini}%
		}
	\caption{Visualization of the work flow for the extraction of the acinar volumes on a rat lung sample extracted at day 60: %
		\subref{subfig:sample}: \threed visualization of sample. For this sample, we stacked three wide-field scans on top of each other, increasing the field of view nine-fold compared to a classic scan at TOMCAT. The borders between the three stacked scans are faintly visible as darker lines, but are only visible in this \threed visualization. %
		\subref{subfig:airway segment} Extracted airway segment (green) superimposed on the sample. Using a threshold based region growing algorithm, we extracted conducting airways inside the sample. The red discs shown are the so-called manhole covers, which were semiautomatically placed and used as segmentation stoppers for the region growing. %
		\subref{subfig:extracted acini} Extracted acini (yellow). Multiple extracted acini are shown superimposed over the sample in \threed. For each such acinus we recorded the volume.%
		}
	\label{fig:workflow}
\end{figure}

The volume of the single acini was calculated by multiplying the amount of segmented voxels with the voxel volume, tabulated in Excel-Files and prepared for analysis. The segmented acini were also exported to single \href{https://secure.wikimedia.org/wikipedia/en/w/index.php?title=Digital_Imaging_and_Communications_in_Medicine&oldid=415023605}{DICOM} stacks to facilitate further processing in either MeVisLab, \href{http://rsbweb.nih.gov/ij/}{ImageJ}~\cite{Abramoff2004} or the \href{http://stepanizer.com/}{STEPanizer}~\cite{Tschanz2011}, as described later on.

\section{Stereological Analysis}
\begin{itemize}
	\item Quick rundown on Stereology, including�“how to do it correctly”/ATS guidelines by \citet{Hsia2010}.
	\item STEPanizer; web based tool, counting with line pairs and automatic tabulation in Excel-Files, as shown in \autoref{fig:STEPanizer}.
	\item Analysis for Volume and Surface, comparison of MeVisLab and STEPanizer-data.
\end{itemize}
\subsection{STEPanizer}
Explain the STEPanizer a wee bit.

\subsection{Comparison of Volumes from MeVisLab with STEPanizer}
As mentioned above, the volume of the single acini was calculated from the amount of segmented voxels multiplied with their size. To check these volumes against a gold standard method, we manually and stereologically assessed the volume of single acini. For this analysis, we prepared datasets in which we combined the segmented acinus with the corresponding region of interest from the original tomographic dataset, as seen in the background of \autoref{fig:STEPanizer} and exported these regions of interest as DICOM files. With a MATLAB script we padded the volumes to a square format, extracted slices in a systematically random way, wrote them to the disk as JPG sequence and analyzed these slices with the STEPanizer. This made it possible to relate the automatically calculated volumes from MeVisLab to an accepted and proven method.

\renewcommand{\imsize}{\linewidth}
\begin{figure}
	\centering
	\includegraphics[width=\imsize]{img/CountingWindowSTEPanizer}
	\caption{STEPanizer counting window, the scalebar in the lower left of the slice is \SI{100}{\micro\meter} long.}
	\label{fig:STEPanzier}
\end{figure}

\section{Results}\label{sec:Results}
For each day and sample we extracted multiple acini with the Manhole Cover-method described in \autoref{sec:manholecovers}. In total we extracted 980 acini for 22 animals, an average of 45 acini per animal. A summary of the data from the extracted acinar volumes is shown in \autoref{tab:summary}. The data has been analyzed using R (Version 2.12.1 (2010-12-16)~\cite{R} in RStudio (\url{http://rstudio.org/}, Version 0.94.84).

\begin{figure}
	\centering
	\begin{tikzpicture}
		\begin{axis}[%
			only marks,
			legend pos=south east,
			ymin=0,
			xlabel=Acinus,
			ylabel={normalized Volume}
			]
			\addplot
				coordinates {
					(1,100)
					(2,100)
					(3,100)
					(4,100)
					(5,100)
					(6,100)
					(7,100)
					(8,100)
					(9,100)
					(10,100)
					(11,100)
					(12,100)
					(13,100)
				};
			\addplot
				coordinates {
					(1,129.8240965)
					(2,89.38977694)
					(3,113.2250652)
					(4,112.2147492)
					(5,110.0954279)
					(6,102.6617113)
					(7,102.4327191)
					(8,94.63869341)
					(9,187.1668149)
					(10,106.5371783)
					(11,159.3668907)
					(12,113.1695745)
					(13,121.3253942)
				};
			\addplot[yellow, mark=square*]
				coordinates {
					(1,129.8240965)
					(9,187.1668149)
					(11,159.3668907)					
				};
			\legend{MeVisLab, STEPanizer}				
	\end{axis} 
	\end{tikzpicture}
	\caption{Volume of multiple acini estimated with different Methods. The volumes have been normalized to the volume obtained with MeVisLab. %
	Yellow acini are {\color{red} TEMPORARILY} shown in \autoref{fig:mevisvsstep}.}
	\label{fig:VolumeMeVisVsSTEPanizer}
\end{figure}

\renewcommand{\imsize}{0.333\linewidth}
\begin{figure}
	\centering
	\subfloat[Acinus 02, Slice 27, Vol. \SI{130}{\percent}]{%
		\includegraphics[width=\imsize]{d:/SLS/2009f/mrg/R108C60Dt-mrg/acinus02/voxelsize1.48-every6slice/R108C60Dt-mrg-acinus02-27.jpg}%
		}%
	\subfloat[24, 46, \SI{187}{\percent}]{%
		\includegraphics[width=\imsize]{d:/SLS/2009f/mrg/R108C60Dt-mrg/acinus24/voxelsize1.48-every6slice/R108C60Dt-mrg-acinus24-46.jpg}%
		}%
	\subfloat[32, 52, \SI{159}{\percent}]{%
		\includegraphics[width=\imsize]{d:/SLS/2009f/mrg/R108C60Dt-mrg/acinus32/voxelsize1.48-every6slice/R108C60Dt-mrg-acinus32-52.jpg}%
		}%
	\caption{{\color{red} TEMPORARY} Illustrative slices of the acini (02, 24 and 32) that show large differences in volume between MeVisLab and STEPanizer. The “bad” segmentation is the reason why we have much larger volumes with the STEPanizer than with MeVisLab.}
	\label{fig:mevisvsstep}
\end{figure}

\begin{itemize}
	\item Show volumes of different acini with MeVisLab and STEPanizer
	\item What about shrinkage?
	\item Comparison with other data?
	\item What about the surface? What about the “Erithrocytes”?
	\item Alveolar number, how can we bring in this data?
\end{itemize}

\subsection{Acinar Volumes}
% from R
%> summary(TotalVolumes)
%       V1                  V2                  V3                  V4                  V5           
% Min.   :4.934e-04   Min.   :1.330e-03   Min.   :1.762e-03   Min.   :2.442e-03   Min.   :2.039e-03  
% 1st Qu.:4.100e-03   1st Qu.:6.502e-03   1st Qu.:1.573e-02   1st Qu.:3.892e-02   1st Qu.:2.742e-02  
% Median :8.597e-03   Median :1.124e-02   Median :3.188e-02   Median :6.950e-02   Median :5.321e-02  
% Mean   :1.178e-02   Mean   :1.398e-02   Mean   :4.015e-02   Mean   :8.069e-02   Mean   :6.707e-02  
% 3rd Qu.:1.805e-02   3rd Qu.:1.977e-02   3rd Qu.:5.801e-02   3rd Qu.:1.099e-01   3rd Qu.:1.038e-01  
% Max.   :3.908e-02   Max.   :3.855e-02   Max.   :1.202e-01   Max.   :2.109e-01   Max.   :2.103e-01  
% NA's   :7.980e+02   NA's   :7.820e+02   NA's   :7.970e+02   NA's   :7.990e+02   NA's   :8.440e+02  
%> 
\begin{table}[h]
	\centering
	\caption{Summary of the extracted Data.}
	\begin{tabular}{rccccc}
		\toprule
		Day & Animals & Median [\si{\micro\litre}] & Mean [\si{\micro\litre}] & extracted Acini\\
		\midrule
		4	& 5 & \num{8.597e-03} & \num{1.178e-02} & 202 \\
		10	& 5 & \num{1.124e-02} & \num{1.398e-02} & 218 \\
		21	& 4 & \num{3.188e-02} & \num{4.015e-02} & 203 \\
		36	& 4 & \num{6.950e-02} & \num{8.069e-02} & 201 \\
		60	& 4 & \num{5.321e-02} & \num{6.707e-02} & 156 \\
		\bottomrule
	\end{tabular}
	\label{tab:summary}
\end{table}

\subsection{Acinar Number and Volume from Sébastien}
\begin{table}[h]
	\centering
	\caption{Acinar Number and mean Volume from Sébastien}
	\begin{tabular}{ccc}
	\toprule
	Day & Number of Acini & mean Volume [\si{\micro\litre}]\\
	\midrule
	4	& 500	& 0.09 \\
	21	& 660	& 0.28 \\
	60	& 600	& 0.8 \\
	\bottomrule
	\end{tabular}
	\label{tab:data sebastien}
\end{table}

\subsection{Comparison with Gold-Standard STEPanizer}

\subsection{Alveolar Number}
\begin{itemize}
	\item Alveolar number from Liliane?
	\item Alveolar number counts from Eveline with STEPanizer?
\end{itemize}

\section{Discussion}\label{sec:Discussion}
\begin{itemize}
	\item Manhole Covers are rad!
	\item We extracted nearly 1000 single acini, at the moment only show data from day 60
	\item Volumes of MeVisLab are comparable with STEPanizer. Why are they not perfectly comparable, discuss!
	\item We can do more with this method
	\begin{itemize}
		\item Surface
		\item Number of Alveoli
		\item All related to SINGLE acinus
	\end{itemize}
\end{itemize}

\section{Contributions}
\begin{itemize}
	\item David Haberthür helped with rat lung sample extraction, data acquisition and analysis of the acinar volume, creation of the MeVisLab networks for extraction of the acini, analysis of the data in the STEPanizer statistical analysis and main writer of the manuscript.
	\item Sébastien Barré was helping with preparations of the samples, data acquisition,  preprocessing and analysis of the datasets.
	\item Lilian Salm: Alveolar numbers
	\item Marco Stampanoni: TOMCAT
	\item Johannes C. Schittny: Group head, grant writer and lung extractor.
\end{itemize}

\section{Acknowledgments}
We thank Federica Marone, Beamline Scientist at TOMCAT for the long-standing great support at the Beamline and implementation of the reconstruction of the merged wide-field scanning projections on the TOMCAT cluster. Christoph Hinterm\"{u}ller, former member of the TOMCAT group, and now Project Management \& Research at \href{http://gtec.at/}{g.tec medical engineering GmbH} also helped us during countless shifts at the beamline. He also helped with the first implementation of wide-field scanning at TOMCAT. Bernd Pinzer, Post-Doc at TOMCAT adopted the support of our group from Chris and also does great job. Milo Hindennach, MeVis Developer at Fraunhofer MEVIS provided the \href{http://www.mevis-research.de/cgi-bin/discus/board-auth.cgi?lm=1282233250&file=/839/11760.html}{Manhole cover module in MeVisLab}, which is still at the core of the MeVisLab-network used in this publication. We thank Mohammed Ouanella, our former lab and Eveline Yao, our current lab technician for expert technical assistance in the lab\todo{Oder ist Eveline auch auf der Autorenliste?}.

The handling of animals before and during the experiments, as well as the experiments themselves, were approved and supervised by the Swiss Agency for the Environment, Forests and Landscape and the Veterinary Service of the Canton of Bern, Switzerland.

This work has been funded by the grants 3100A0-109874 and 310030-125397 \todojcs{Still correct?} of the Swiss National Science Foundation.

\bibliographystyle{unsrtnat}
\bibliography{../references}

\end{document}